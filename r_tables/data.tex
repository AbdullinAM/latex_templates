\section{Цель работы}
Овладение методикой расчёта и экспериментальные исследования основных параметров однокаскадных транзисторных усилителей, получение навыков настройки их режимов и снятия частотных характеристик усилителей.
\section{Исходные данные и расчёты}
\begin{multicols}{2}
{\setlength{\parindent}{0cm}
Транзистор МП-39

$E_0 = 8$ В

$U_{кэА} = 4$

$R_к = 3.9$ кОм

$R_н = 2$ кОм

$r_б = 220$ Ом

$C_{p1} = 0.22$ мкФ

$C_{p2} = 0.47$ мкФ

$h_{21} = 12$

$C_к = 60$ нФ

$r_б = 220$ Ом

$f_{\alpha} = 0.5$ мГц

$h_{11э} = 820$ Ом

$U_{бэ} = 0.6$ В

\vspace{1mm}
$K_{u0} = \dfrac{U_н}{e_c} = \dfrac{-h_{21}R_кR_н}{h_{11э}(R_{к}+R_{н})} = 19.34$

\vspace{2mm}
$K_{i0} = -h_{21}\dfrac{R_1}{R_1+h_{11э}}\dfrac{R_к}{R_к+R_н} =$

\vspace{1mm}
$= 12\dfrac{86.5\cdot10^3}{86.5\cdot10^3+820}\dfrac{3.9\cdot10^3}{(3.9+2)\cdot10^3} = 7.86$

%$R_{вх} = r_б + r_э(h_{21}+1) = 8500$ Ом
\vspace{2mm}
$R_1 = \dfrac{h_{21}(E_0-U_{БЭ})R_к}{E_0-u_{кэА}} = \dfrac{12\cdot(8-0.6)\cdot3.9\cdot10^3}{8-4} = \linebreak = 86.58$ кОм

$R_{1.практ} = 100 кОм + 120 кОм$

$20\lg K_{u0} = 25.7$

$a = 1.2$ мкс

$f_в = \dfrac{1}{2 \pi a} = 122 488$ Гц

}
\end{multicols}
\section{Вывод}
Экспериментальным путём было выяснено, что у транзисторных усилителей коэффициент усиления непостоянен и зависит от частоты входного сигнала. Причина этого - наличие двух емкостей в схеме, на входе и на выходе, а также физические свойства транзистора.

Теоретические частоты перегиба для высоких и низких частот ($f_в, f_{1н}, f_{2н}$) оказались больше практических по причине низкой точности используемых компонентов.

Теоретический коэффициент усиления по напряжению ($K_{u0}=19.34$) оказался ниже практического по той же причине. 

Таким образом можно сделать вывод, что данная схема усилителя имеет проблемы с предсказуемостью параметров конкретного экземпляра по причине большого разброса параметров компонентов, поэтому для применения в реальной жизни требуется использовать улучшенные схемы усилителей (например, с отрицательной обратной связью).

Значительное падение коэффициента усиления при небольшой величине входного сигнала происходит из-за увеличения влияния шумов.
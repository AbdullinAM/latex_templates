% Реферат
\section*{\centering Реферат}

Петров В.Д. Тема работы: Курсовой проект по дисциплине "Название предмета".- СПб.: СПбПУ, \the\year. \begin{NoHyper}{\pageref{LastPage}}\end{NoHyper}  с, рис. - \totalfigures{}, табл. - \totaltables{}, листинги - \total{listnum}, библиогр. - \total{citnum} назв.\\[1cm]

ПЯТЬ, КЛЮЧЕВЫХ, СЛОВ, ИЛИ ФРАЗ\\

Список ключевых слов должен стоять ил 5 или 6. слов (например, МАРКЕТИНГ, ИССЛЕДОВАНИЯ) или коротких выражений (ИЗУЧЕНИЕ УДОВЛЕТВОРЕННОСТИ, РАЗРАБОТКА ПРОДУКТА). Их цель - облегчить поиск опубликованных трудов. Практически это выглядит так: если набрать список ключевых слов в поисковой системе, должна появиться наша пояснительная записка вместе с небольшим количеством работ той же тематики.\\

Реферат - важнейшая часть работы. в нем в нескольких строках излагаются полученные результаты. Поэтому в тексте реферата излагается только самая суть нашей работы (получено... рекомендовано...)‚ без лишних фраз (...имеет большое значение...‚ рассмотрено..., исследовано...).


\thispagestyle{empty} % не нумеровать страницу
\newpage

